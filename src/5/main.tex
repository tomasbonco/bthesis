\chapter{Záver}
V~tejto práci sme najskôr predstavili najpoužívanejšie spôsoby na~nahrávanie obrázkov v~súčasnosti. Následne sme poukázali na~ich nedostatky, ktoré boli základom pre~naše riešenie. Stanovili sme si ambiciózne ciele a~následne sme naše riešenie vytvárali tak, aby ich spĺňalo.
Chceli sme vytvoriť knižnicu, ktorá bude nezávislá od~iných knižníc a~frameworkov. To sa nám použitím iba JavaScriptu poradilo. Vďaka zapuzdreniu, ktoré poskytujú webové komponenty nie je možné, aby tento JavaScript zasahoval do~ostatných častí aplikácie. Takisto implementácia knižnice ako webovej komponenty spôsobuje, že naše riešenie je veľmi jednoduché na~použitie.
Podporuje nahrávanie obrázkov cez~modálne okno, ale aj modálne okno a~rovnako podporuje aj nahrávanie vieacerých obrázkov. Všetko toto je možné spraviť aj cez~mobilné zariadenia. Obrázky je možné nahrávať či bežnou cestou cez~formulár (v elemente \emph{<form>}) alebo cez~\emph{XMLHttpRequest} a~teda bez~refreshu. Užívateľ ale v~oboch prípadoch vidí nahrávaný obrázok ešte pred~odoslaním a~má možnosť upraviť orezanie.
Keďže si uvedomujeme, že použitie elementu \emph{<canvas>} na~zobrazenie obrázka s~možnosťou prístupu až k~jednotlivým pixelom otvára dvere obrovskému množstvu možností (napríklad zoomovanie, detekcia tvári, úprava farebnej škaly, inteligentné orezanie), ktorých implementácia nie je v~naších silách, vymysleli a~popísali sme jednoduchý, zato elegantný spôsob ako tieto rozšírenia písať, použiť a~kombinovať a~to či už globálne alebo pre~konkrétne elementy. V~neposlednom rade sme použitie zdokumentovali a~s celou knižnicou zverejnili v~službe \emph{Github}.

Po predstavedstavení a~rozbore nášho riešenia v~kapitole \ref{sec:solution}, sme ho v~ďalšej kapitole otestovali a~porovnali s~vybranými inými riešeniami, ktoré sú založené na~podobných princípoch. Na~konci tejto kapitoly ešte uvádzame odporúčané použitie a~uvažujeme, akým spôsobom by sa mohlo naše riešenie ďalej posúvať. 

\section{Odporúčané použitie}
\section{Nápady na~vylepšenie}