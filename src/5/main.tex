\chapter{Záver}
V~tejto práci sme najskôr predstavili najpoužívanejšie spôsoby na~nahrávanie obrázkov v~súčasnosti. Následne sme poukázali na~ich nedostatky, ktoré boli základom pre~naše riešenie. Stanovili sme si ambiciózne ciele a~následne sme naše riešenie vytvárali tak, aby ich spĺňalo.
Chceli sme vytvoriť knižnicu, ktorá bude nezávislá od~iných knižníc a~frameworkov. To sa nám použitím iba JavaScriptu podarilo. Vďaka zapuzdreniu, ktoré poskytujú webové komponenty nie je možné, aby tento JavaScript zasahoval do~ostatných častí aplikácie. Takisto implementácia knižnice ako webového  komponentu spôsobuje, že naše riešenie je veľmi jednoduché na~použitie.
Podporuje nahrávanie obrázkov cez~\emph{Drag\&Drop}, ale aj modálne okno a~rovnako podporuje aj nahrávanie viacerých obrázkov. Všetko toto je možné spraviť aj cez~mobilné zariadenia. Obrázky je možné nahrávať bežnou cestou cez~formulár (v elemente \emph{<form>}) alebo cez~\emph{XMLHttpRequest}, a~teda bez~obnovenia stránky. Užívateľ ale v~oboch prípadoch vidí nahrávaný obrázok ešte pred~odoslaním a~má možnosť upraviť orezanie.
Keďže si uvedomujeme, že použitie elementu \emph{<canvas>} na~zobrazenie obrázka s~možnosťou prístupu až k~jednotlivým pixelom otvára dvere obrovskému množstvu možností (napríklad približovanie, detekcia tvárí, úprava farebnej škaly, inteligentné orezanie), ktorých implementácia nie je v~našich silách, vymysleli a~popísali sme jednoduchý, ale elegantný spôsob, ako tieto rozšírenia písať, použiť a~kombinovať, a~to či už globálne alebo pre~konkrétne elementy. V~neposlednom rade sme použitie zdokumentovali a~s celou knižnicou zverejnili v~službe \emph{GitHub}.

Po predstavení a~rozbore nášho riešenia v~kapitole \ref{sec:solution} sme ho v~ďalšej kapitole otestovali a~porovnali s~vybranými inými riešeniami, ktoré sú založené na~podobných princípoch. Na~konci tejto kapitoly ešte uvádzame odporúčané použitie a~uvažujeme, akým spôsobom by sa mohlo naše riešenie ďalej posúvať. 


\section{Prínosy a~odporúčané použitie}

Knižnica umožňuje okamžite zobraziť obrázok bez~potreby serveru. Všetky zmeny sa dejú v~prehliadači, a~teda skúsenosť používateľa nie je zaťažená načítavaním. Odporúčame ju používať na~projekty, kde je známe finálne rozlíšenie zobrazovaného obrázka (napr. blogy, internetové obchody, firemné stránky, profilové obrázky) a~nie je potrebné si uchovávať pôvodný veľký obrázok. Hoc je možné použiť knižnicu aj v~takýchto prípadoch, strácajú sa ďalšie prínosy knižnice -- napríklad skutočnosť, že sa odosiela zmenšená a~orezaná verzia, ktorá tak menej zaťažuje sieť a~v prípade spoplatnených mobilných dát šetrí peniaze používateľa. Tiež sa stráca výhoda, že nie je potrebné implementovať zmenšovanie a~orezávanie obrázka na~serveri. V~prípade odosielania obrázka ako \emph{Base64} reťazca (štandardný spôsob) je dokonca množstvo prenášaných dát asi o~30 \% väčšie.


\section{Nápady na~vylepšenie}

Vzhľadom na~obmedzenia, na~ktoré sme počas vývoja narazili, dúfame, že našu prácu bude možné v~budúcnosti vylepšiť v~nasledujúcich oblastiach:

\begin{description}
	
	\item[Zrkadlenie časti obsahu Custom Elementu do~DOM.] Ako sme v~kapitole \ref{sec:odoslanie-obrazka} uviedli, nezhoda medzi~tvorcami prehliadačov spôsobuje, že v~dobe písania sa už neodporúča na~zrkadlenie využiť element \emph{<content>}, ale ešte sa neodporúča používať element \emph{<slot>}. Po~štandardizovaní štandardu a~následnom zapracovaní do~knižnice by to umožnilo odosielanie viacerých obrázkov vo formulári (elemente \emph{<form>}).
	
	\item[Odosielanie obrázkov ako \emph{Blob}.] Vzhľadom na~to, že ukladanie obrázka ako \emph{Base64} reťazec zväčšuje jeho veľkosť o~30 \%, ideálne by bolo ukladať a~prenášať ho ako \emph{Blob}. To je v~dobe písania tejto práce podporované len v~prehliadači Mozilla Firefox, IE 11 a~Google Chrome ohlásil podporu v~budúcej verzii (50) \cite{Canvas_toBlob}.
	
\end{description}