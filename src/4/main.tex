\chapter{Validácia riešenia}
\section{Podpora mobilných zariadení}

% Android 5.0.2 - LG G2
Naše riešenie sme testovali na~dvoch zariadeniach - LG G2 (Android 5.0.2, natívny prehliadač aj Chrome) a~iPhone 6 (iOS 9.3.1, Safari aj Chrome) a~v oboch prípadoch boli výsledky úspešné. Cieľom bolo overiť, či naše riešenie funguje správne na~dvoch najväčších platformách. Implementácia sa pre~mobilné prehliadače jemne líši. Kým užívatelia na~stolových počítačoch používajú v~prevažnej miere myš, mobilné zariadenia sú dotykové, a~teda bolo potrebné zabezpečiť, aby sa dotyky správne interpretovali - používanie \emph{event.changedTouches[0].pageX} miesto \emph{event.pageX} a~tiež povolenie posúvania obrázka len jedným prstom.

Takisto sme si všimli, že náhrada (\quoted{polyfill}) \emph{webcomponents.js}, ktorú používame, aby sme dosiahli plnú podporu na~dosiaľ nepodporovaných prehliadačoch, nesvytvára pseudotriedu \emph{:host} na~mobilných zariadeniach, preto je potrebné do~CSS manuálne vložiť spôsob, akým sa má element vykresliť, ako uvádza ukážka \ref{sourcecode_host}.

\begin{lstlisting}[label=sourcecode_host,caption=Štýly potrebné pre~správne fungovanie na~mobilných zariadeniach.]
x-cupe, x-cupe-gallery { display: inline-block; }
\end{lstlisting}

\section{Porovnanie s~ďalšími riešeniami}

Vzhľadom k~tomu, že knižníc na~nahrávanie obrázkov je veľa, rozhodli sme sa porovnať tú našu s~vybranými ostatnými. Porovnávali sme spôsob a~rýchlosť nahrávania, spôsob spracovania a~ponúkané možnosti. Všetky knižnice musia upravovať obrázok v~prehliadači.
Testy sme vykonávali s~dvoma obrázkami - s~obrázkom ABSTRACT ($3000\times2000$px; 72dpi; 5,38MB) a~s väčším obrázkom PANORAMA ($24442\times4195$px; 150dpi; 28,1MB).

\begin{description}
	
	\item[http://scottcheng.github.io/cropit/ (A)] (30. 4. 2016)\\
	je jQuery rozšírením, umožňuje nahrávať obrázky pomocou \emph{Drag\&Drop} aj vybraním obrázka cez~modálne okno. Podporuje približovanie (len pomocou posuvnika), a~tiež rotovanie obrázka. Na~vykresľovanie obrázka nepoužíva \emph{<canvas>}, ale \emph{<img>}, s~tým, že keď je potrebné obrázok uložiť, prekreslí vybraný výsek obrázka do~elementu \emph{canvas}, z~ktorého výsledok vyexportuje. Aj napriek tomu, že na~stránke projektu sa uvádza, že je veľmi rýchle aj pri~veľkých obrázkoch, v~našich meraniach trvalo nahranie obrázka PANORAMA 26 sekúnd, a~následné posúvanie nebolo možné, nakoľko sa obrázok prekresľoval niekoľko sekúnd.
	
	%25.99
	
	\item[http://foliotek.github.io/Croppie/ (B)] (30. 4. 2016)\\
	je ďalšie jQuery rozšírenie, ktoré slúži na~úpravu obrázkov. Podporuje približovanie aj stredným tlačidlom myši. Samotná knižnica nepodporuje nahrávanie obrázkov. Umožňuje úpravu obrázkov už nahraných, a~to takým spôsobom, že ich vykreslí do~elementu \emph{<img>}, ktorým umožňuje posúvať a~pri uložení prekreslí výrez obrázka do~canvasu, ktorý vyexportuje. Vzorové použitie však ukazuje aj prípad, kedy knižnica dokáže spracovať nahrané dáta. V~tom prípade celý obrázok vykreslí do~elementu \emph{<canvas>}. Následne s~ním pracuje rovnako ako v~prvom prípade s~elementom \emph{<img>}. Nahrávanie obrázka PANORAMA trvalo priemerne 13 sekúnd (od odoslania po~zobrazenie obrázka), ale na~jeho plynulé posúvanie bolo potrebné čakať ešte zhruba minútu a~pol. Myslíme si, že je to spôsobené tým, že vytvorený element \emph{<canvas>} mal rovnaké rozlíšenie ako originálny obrázok, čo značne predlžovalo dobu vykreslenia v~prehliadači.
	
	%15.34
	%2.06.11
	
	\item[http://andyvr.github.io/picEdit/ (C)] (30. 4. 2016)\\
	umožňuje nahrávanie obrázka cez~modálne okno. Obrázok následne vykreslí ako pozadie elementu \emph{<div>}. Približovanie mení parametre\footnote{používajú sa CSS parametre \emph{background-size} a~\emph{background-position}}, akými sa toto pozadie zobrazuje. Následné ukladanie prebieha opäť cez~prekreslenie do~elementu \emph{<canvas>}, z~ktorého je výsek vyexportovaný. Nahranie obrázka PANORAMA bolo celkom rýchle, avšak posúvanie nebolo možné a~približovanie trvalo dlhú dobu.
	
	
	\item[http://codecanyon.stbeets.nl/ (D)](30. 4. 2016)\\
	je platené (11\$) rozšírenie jQuery umožňuje nahrávanie obrázkov aj cez~modálne okno, aj cez~\emph{Drag\&Drop}. Nahraný obrázok vykresľuje v~elemente \emph{<img>}, ktorým umožňuje posúvať, a~následné uloženie obrázka spôsobí prekreslenie do~elementu \emph{<canvas>}. Uloženie je potrebné spraviť ručne. Pri~nahratí obrázka ABSTRACT posúvanie mierne sekalo, pri~nahratí obrázka PANORAMA nebolo možné.

\end{description}

Tieto vybrané knižnice sme porovnali v~nasledujúcich vlastnostiach:
\begin{enumerate}
	\item Závislosti na~knižniciach tretích strán.
	\item Podpora odosielania obrázka cez~formulár (element \emph{<form>}).
	\item Nahrávanie viacerých obrázkov.
	\item Podpora nahrávania cez~\emph{Drag\&Drop}.
	\item Možnosť stanoviť orezanie (napríklad posúvaním obrázka).
	\item Podpora približovania.
	\item Čas nahratia obrázka ABSTRACT\footnote{Test prebiehal na~ultrabooku ASUS Zenbook UX31A (Intel i7-3517U, 4GB RAM, Microsoft Windows 10, Chrome 50. Časy boli odčítané z~videa, ktoré bolo nahrávané počas nahrávania obrázkov.) }.
	\item Čas nahratia obrázka PANORAMA\footnotemark[\value{footnote}].
\end{enumerate}

\begin{table}[!htb]
	\centering
	\begin{tabular}{@{}|l|c|c|c|c|c|@{}}
	\toprule
		  		   & \multicolumn{1}{l|}{Naša knižnica} & \multicolumn{1}{l|}{A} & \multicolumn{1}{l|}{B} & \multicolumn{1}{l|}{C} & \multicolumn{1}{l|}{D} \\ \midrule
	1              & žiadne                & jQuery                 & jQuery                 & žiadne                 & jQuery                 \\ \midrule
	2 		       & áno                   & nie                    & nie                    & nie                    & áno                    \\ \midrule
	3  			   & áno                   & nie                    & nie                    & nie                    & nie                    \\ \midrule
	4			   & áno                   & nie                    & nie                    & nie                    & áno                    \\ \midrule
	5		       & áno                   & áno                    & áno                    & áno                    & áno                    \\ \midrule
	6              & áno                   & áno                    & áno                    & áno                    & áno                    \\ \midrule
	7 			   & 0,9s                  & 3,6s                   & 2,5s                   & 1,3s                   & 2,9s                  \\ \midrule
	8   		   & 3,6s                  & 23,6s                  & min. 17,6s 			 & 7,6s                   & 15,6s                 \\ \bottomrule
	\end{tabular}
	\caption{Porovnanie našej knižnice s~inými vybranými riešeniami.}
	\label{my-label}
\end{table}

Nepodarilo sa nám nájsť riešenie, ktoré by fungovalo obdobne ako naše (používalo po~celú dobu len element \emph{<canvas>}). Každé z~vybraných riešení k~problematike nahrávania a~zobrazovania obrázka pristupuje inak. Žiadne z~vybraných riešení nebolo schopné vysporiadať sa so~skutočne veľkým obrázkom, zatiaľčo naše riešenie ho nie len nahralo v~rekordne krátkom čase, ale zároveň hneď umožňovalo aj jeho plynulú manipuláciu.