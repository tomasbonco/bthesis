\chapter{Vytvorenie knižnice na~nahrávanie obrázkov}
\section{Ciele}

Ciele, ktoré sme stanovili pri~vytváraní knižnice majú za~cieľ maximalizovať pohodlie užívateľa, minimalizovať zaťaženie pre~tvorcov stránky, sprístupniť knižnicu všetkým. Preto sme stanovili, že knižnica by mala:

\begin{itemize}
	\item byť nezávislá od~knižníc/frameworkov tretích strán, napr. \emph{jQuery},
	\item podporovať nahrávanie či už zvolením obrázka cez~modálne okno alebo aj technológiou \emph{Drag&Drop}, 
	\item byť jednuduchá na~implementovanie
	\item byť dobre rozšíriteľná o~nové funkcie
	\item podporovať nahrávanie bez~refreshu
	\item podporovať mobilné zariadenia
	\item podporovať nahrávanie viacerých obrázkov
	\item byť dobre zdokumentovaná
\end{itemize}


\section{Riešenie}
\subsection{Predstavenie}
Najskôr sme plánovali využiť \emph{jQuery}, ale kvôli splneniu prvého cieľa (byť nezávislí od~knižníc/frameworkov tretích strán) sme sa rozhodli využiť \emph{Custom Element}. Vytvorili sme jeden, ktorý slúži na~úpravu obrázka, \emph{<x-cupe>} a~jeden na~podporu multiuploadu - \emph{<x-cupe-gallery>}. Oba vychádzajú z~rovnakého HTML, pričom \emph{<x-cupe-gallery>} interne využíva \emph{<x-cupe>}, preto sa v~tejto kapitole zameráme na~element \emph{<x-cupe>}. Nahrávanie viacerých obrázkov je bližšie popísané v~kapitole 3.5.

Scénar nahrávania môže vyzerať takto:
\begin{enumerate}
	\item Užívateľ buď pomocou Drag&Drop nahraje obrázok alebo klikne na~element a~následne si z~modálneho okna zvolí obrázok
	\item Obrázok sa zmenší, oreže a~zobrazí
	\item Pokiaľ užívateľ nie je spokojný s~obrázokom, pokračuje prvým bodom
	\item Pokiaľ užívateľ nie je spokojný s~orezaním, držaním myšy posúva obrázok, dokým nie je spokojný
	\item Keď je užívateľ spokojný s~obrázkom aj jeho orezaním odošle formulár, a~až vtedy sa zmenšaná a~orezaná verzia nahraje na~server
\end{enumerate}

Hoc sa v~scenári spomína klikanie, knižnica podporuje aj dotykové zariadenia a~teda všetky úkony je možné vykonať dotykom. Interne obe elementy využívajú nasledujúce HTML:

\begin{lstlisting}[language=HTML]
<canvas></canvas>
<input type="file" style="display: none">
<input type="text" style="display: none">
\end{lstlisting}

Element \emph{<canvas>} sa stará o~zobrazenie obrázka, a~na prácu s~ním sme vytvorili pomocnú triedu \emph{XCupeCanvasElement}, pomocou ktorej vieme rýchlo zmeniť rozmery. Keď užívateľ klikne na~\emph{<x-cupe>}, element vytvorí nový klik na~\emph{<input type="file"\textgreater}, čo spôsobí, že sa otvorí modálne okno a~užívateľ si môže zvoliť obrázok. Pre~zjednodušenie práce s~ním sme vytvorili triedu \emph{XCupeInputFileElement}. Do~\emph{<input type="text"\textgreater} sa vloží textová reprezentácia obrázka, čo umožňuje jeho odoslanie cez~POST. Opäť, aj pre~tento element sme vytvorili pomocnú triedu \emph{XCupeInputFileElement}. Detailné popísanie, ako knižnica funguje vrátane využitia jednotlivých elementov je ďalej popísané v~sekcií 3.4.

\subsection{Inštalácia}

Použitie \emph{Custom Elementu} nám umožnuje skutočne jednoduchý spôsob inštalácie, kde sa najskôr načítajú zdrojové súbory knižnice:
\begin{enumerate}
<link rel="import" href="dist/x-cupe.html">
\end{enumerate}

Následne stačí už len na~príslušnom mieste v~kóde vytvoriť element:
\begin{enumerate}
<x-cupe></x-cupe>
\end{enumerate}

V prípade nahrávania viacerých obrázkov:
\begin{enumerate}
<x-cupe-gallery></x-cupe-gallery>
\end{enumerate}

// TODO: schéma prepojenia jednotlivých elementov a~knižníc. Bolo by to vhodné?

\section{Použité technológie, služby a~postupy}
\subsection{Custom Element a~ShadowDOM}
\subsection{Canvas}

Canvas je nový tág \emph{HTML5}, ktorý umožňuje pomocou JavaScriptu vykresľovať grafy, úpravovať fotografií, animovať a~dokonca aj spracovať videa. Využíva ho aj \emph{WebGL} na~hardvérovo akcelerovanú 3D grafiku na~webe\cite{MDN_Canvas}.

V našej práci ho využívame na~zobrazovanie obrázka. Samotná knižnica už obsahuje potrebné nástroje na~vkladanie a~zmenšovanie obrázka, čo nám uľahčuje prácu a~zároveň nám \emph{<canvas>} umožňuje prístup k~jednotlivým pixelom, čo odcenia najm tvorcovia pokročilých rozšírení (napr. detekovať nevyužívanú plochu obrázka alebo prevod do~čierno-bielej verzie). Keď užívateľ presúva obrázok, kvôli orezaniu, prekresľujeme daný výrez do~\emph{<canvas>} pomocou \emph{requestAnimationFrame}\cite{MDN_RequestAnimationFrame}, pre~maximálnu plynulosť.

\subsection{Drag&Drop}
\subsection{File Reader}

File Reader umožňuje čítať obsah súborov vo webovom prostredí. Tie môžu byť načítané buď pomocou tágu \emph{<input>} alebo pomocou \emph{Drag&Drop} (viď. 3.3.3)\cite{MDN_FileReader}\footnote{Citovaný MDN udáva ešte aj HTMLCanvasElement.mozGetAsFile(), ktorý je neštandardizovaný a~v dobe písanie tejto práce podporovaný iba prehliadačom Firefox}.

Použili sme ho práve pre~prečítanie zvolených súborov, ktoré sme následne zobrazili v~\emph{<canvas>} (viď 3.3.2).

\subsection{Promise}

Promise je objekt určený pre~asynchrónne výpočty a~výpočty naplánované na~neskôr (napríklad cez~\emph{setTimeout} alebo \emph{setInterval}). Nevracia priamo metódu, ale vracia "prísľub", že niekedy v~budúcnosti hodnota bude dostupná\cite{MDN_Promise}. V~silno asynchrónnej aplikácií zabraňuje tzv. "callback hell", kde množstvo callbackov bráni zrozumiteľnosti a~ladeniu kódu.

Promise sme používame miesto predávania callbackov, predovšetkým pri~čítaní užívateľovho obrázku.

\subsection{TypeScript}

TypeScript je typová nadstavba JavaScriptu, ktorá sa prekladá do~JavaScriptu. Tým, že sa rýchlo vyvíja okrem~typovosti prináša aj funkcie \emph{ECMAScript 2015 (ES6)}, čo umožňuje progrmátorom používať funkcie, ktoré v~čase písania tejto bakalárskej práce ešte nie sú implementované v~prehliadačoch.

Naša knižnica využíva TypeScript pretože umožňuje rýchlejšie odhaľovať chyby, vedie k~písaniu rozhraní, čo môže značne pomôcť tvorcom rozšírení a~tiež kvôli podpore \emph{ECMAScript 2015}. Navyše jeho vlastnosť, že sa prekladá do~JavaScriptu nijak neobmedzuje tvorcov rozšírení alebo jeho celkové použitie.

\subsection{Mocha}
\subsection{GitHub}

Počas tvorby sme používali Git a~výsledok našej práce sme uverejnili v~službe GitHub - webovej službe na~správu Git repozitárov. // TODO: pokračovať; prečo sme to spravili?

\section{Postup nahrávania a~nastavenia}

V tejto sekcií sa budeme samotnnému procesu nahratia obrázku, od~jeho zvolenia, úpravy po~odoslanie na~server.

\subsection{Inicializácia knižnice}
\subsection{Zvolenie obrázka}

Užívateľ môže obrázok zvoliť dvoma spôsobmi - výberom z~modálneho okna alebo pomocou \emph{Drag&Drop}. V~prípade nahrávania obrázka cez~modálne okno, užívateľ musí kliknúť na~\emph{<x-cupe>} element, do~ktorého chce obrázok nahrať. Interne sa tento klik prevedie na~\emph{<input type="file"\textgreater} a~to spôsobí otvorenie modálneho okna. Po~tom, čo si užívateľ zvolí obrázok, je tento súbor pomocou \emph{File Reader} (viď. 3.3.4) prečítaný a~pripravený na~ďalšie spracovanie. V~prípade, že chceme tento spôsob nahrávania zakázať, stačí nastaviť atribút \emph{allow-select} na~hodnotu \emph{false}.

V prípade použitia Drag&Drop a~teda označenie obrázka, jeho presunutie ponad cieľový element a~následne pustenie myši/oddialenie prsta od~dotykovej plochy, je postup podobný. Element (<x-cupe>) vie pomocou zachytenia udalosti \emph{drop} zachytiť presúvaný obrázok a~dalej ho rovnako ako pri~modálnom okne pomocou \emph{File Reader} prečítať. Ak chceme tento spôsob zakázať, je potrebné nastaviť atribút \emph{allow-drop} na~hodnotu \emph{false}.

\subsection{Zmenšenie a~orezanie}

Po zvolení obrázka spravidla prebieha jeho zmenšenie a~orezanie. Pre~túto činnosť je potrebné vedieť rozmery nahrávaného obrázka a~tiež rozmery výsledného obrázka, pričom sú implementované tieto scenáre:

\begin{itemize}
	\item[Nahrávaný obrázok je väčší a~mal by sa zmenšiť; orezávanie je zapnuté] \hfill \\
	Očakávame, že práve tento scénar bude najvyužívanejší. V~tomto scenári sa obrázok pomerne zmenší tak, aby pomerne kratšia strana nahrávaného obrázka bola zhodná s~príšlusnou stranou finálneho obrázka. To je z~toho dôvodu, aby užívateľ mohol hýbať obrázkom v~ose, ktorá je pomerne dlhšia (a teda v~tej, v~ktorej má posúvanie zmysel). Pre~lepšie porozumienie uvádzame aj obrázok. // TODO obrázok
	\item[Nahrávaný obrázok je väčší a~mal by sa zmenšiť; orezávanie je vypnuté] \hfill \\
	V~tomto prípade sa snažíme vložiť celý nahrávaný obrázok do~finálneho obrázka. Dosiahneme to pomerným zmenšením obrázka tak, aby pomerne dlhšia strana bola zhodná s~príslušnou stranou finálneho obrázka.
	\item[Nahrávaný obrázok je menší a~mal by si zväčšiť; orezávanie je zapnuté] \hfill \\
	Cieľom našej knižnice je, aby spravca stránky vždy dostal z~prehliadača taký obrázok, aký požaduje. Nemusí potom upravovať obrázok na~serveri. To však znamená, že môže nastať prípad, že užívateľ zvolí menší nahrávaný obrázok, než je požadovaný finálny obrázok. V~tom prípade prebieha spracovanie obrazu rovnako, ako v~prvom scenári, len miesto zmenšovania sa obrázok zväčšuje.
	\item[Nahrávaný obrázok je menší a~mal si zväčšiť; orezávanie je vypnuté] \hfill \\
	Postupujeme rovnako ako v~prípade s~väčším obrázkom. Miesto zmenšovania však obrázok zväčšujeme.
	\item[Finálny obrázok má len jeden fixný rozmer] \hfill \\
	Tento scenár je vhodný, ak vytvárame napr. obrázkove galérie, pričom chceme, aby obrázky mali pevnú šírku a~ľubovoľne dlhú dĺžku. Ak chceme variabilnú šírku, je potrebné nastaviť atribút \emph{width} elementu \emph{<x-cupe>} na~hodnotu \emph{-1}. V~prípade, že chceme variabilnú dĺžku, je potrebné obdobne nastaviť parameter \emph{height} na~hodnotu \emph{-1}. V~tomto scenári sa obrázok pomerne zmenší podľa fixnej príslušnej strany finálneho obrázka. Druhá strana finálneho obrázka sa nastaví pomerne zmenšej dĺžke finálneho obrázka. V~tomto scenári nie je možné obrázok orezávať.
	\item[Finálny obrázok nemá fixný rozmer] \hfill \\
	Tento scénar je vhodný, ak chceme aby užívatelia mohli zaslať obrázok v~ľubovoľných rozmeroch a~zároveň ho pred~odoslaním mohli vidieť. Finálny obrázok bude mať rozmery nahrávaného obrázka a~ten teda nie je zmenšovaný ani orezávaný. Pre~vytvorenie tejto situácie treba nastaviť aj atribút \emph{width} aj atribút \emph{height} na~hodnotu \emph{-1}.
	
\end{itemize}


Obrázok sa orezáva posúvaním obrázka (viď 3.4.4) a~základným nastavením je orezanie na~stred. To sa dá zmeniť atribútom \emph{align}, pričom valídne hodnoty pozostávajú z~kombinácie slov \emph{left}, \emph{right}, \emph{top}, \emph{bottom} a~\emph{center}, napríklad: \emph{<x-cupe align="left bottom"\textgreater}. Pri~zadaní iba jednej osi pre~zarovnanie sa automaticky predpokladá, že druhá sa má zarovnať na~stred. Orezanie sa dá úplne vypnúť a~to nastavením atribútu \emph{crop} na~hodnotu \emph{false}. Vnútorne knižnica pracuje tak, že sa textové hodnoty prerátajú na~odsadenie obrázku zľava a~zhora, ale obrázok sa nijako nemodifikuje. To nastáva až pri~vykreslení.

Takisto sme si všimli, že ak zmenšíme veľmi veľký obrázok, jeho kvalita je nízka. Oproti pôvodnému obrázku je zdanlivo ostrejší a~to je spôsobené nevyhovujúcim spôsobom prekresľovania obrázka, tzv. \emph{downsampling}-om. Problém je, že prehliadač musí z~plochy napríklad 4x4 pixely vytvoriť iba jeden pixel a~aby to spravil najrychlejšie, jeden z~nich vyberie, pričom sa neberie do~úvahy kontext. Preto sme aplikovali jednoduchý algoritmus, ktorý obrázok zmenší postupne. Počet prekreslení je možné nastaviť atribútom \emph{quality} a~jeho predvolená hodnota je 3. Čím vyššia táto hodnota je, tým dlhšie trvá proces od~zvolenia obrázka po~jeho vykreslenie. Keď je táto hodnota príliš nízka (napríklad 1), tak je slabšia kvalita. Zmenšený obrázok sa uloží, aby pri~jeho ďalšej manipulácií ho nebolo potrebné opäť zmenšovať.

\subsection{Posúvanie obrázka}

Pokiaľ je povolené orezanie obrázka, je možné toto orezanie meniť posúvaním obrázka. Aby sme odlíšili kliknutie, kedy má vyskočiť modálne okno s~možnoťou výberu obrázka a~samotné posúvanie, nastavili sme limit, počas ktorého musí užívateľ držať myš stlačenú na~100ms. Rovnaký limit platí aj pre~dotykové zariadenia. Po~spustení udalosti presúvania sa vyrátava súčastná pozícia myšy a~porovnáva sa s~pozíciou na~ktorej začalo presúvanie. Rozdielom týchto hodnôt získavame informáciu o~tom, ako máme posunúť obrázok. Tento rozdiel sa následne priráta k~orezaniu obrázka. Nakoniec už zmenšený obrázok s~novým nastavením orezania vykreslíme.

// TODO: Prečo neprebieha ukladanie ak zvyčajne prebieha, ak prebieha? 

\subsection{Odoslanie obrázka}

\section{Nahrávanie viacerých obrázkov}
\section{Rozšíriteľnosť}
\section{Spracovanie obrázka na~PHP serveri}
\section{Dokumentácia}