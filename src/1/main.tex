\chapter{Súčasné metódy nahrávania obrázkov}

Aby sa dali veci robiť lepšie a~efektívnejšie, je najprv potrebné preskúmať už vytvorené spôsoby a~uvedomiť si ich nedostatky. Zamedzí sa tak opakovaniu chýb.
  

\section{HTML tág <input\textgreater}

Historicky prvým spôsobom je nepárový tág \emph{<input type="file"\textgreater}. Objavil sa už v~roku 1997, keď konzorcium W3C vydalo štandard \emph{HTML 3.2} \cite{html32}. Umožňuje nahrávanie súborov, a~teda aj obrázkov, ako súčasť zaslaného formulára (musí sa vyskytovať uprostred tágu <form\textgreater). Scénar nahrávania prebieha nasledovne:

\begin{figure}[!hbt]
	\centering
	\begin{tikzpicture}[node distance=.8cm, start chain=going right,]
		\node[punktchain, join] (input) {\emph{<input>} element};
		\node[punktchain, join] (modal) {Zvolenie obrázka z~modálneho okna};
		\node[punktchain, join, on chain=going below] (form-sending) {Odoslanie formulára};
		\node[punktchain, join, on chain=going left] (server-processing) {Spracovanie na~serveri};
	\end{tikzpicture}
	
	\caption{Schéma nahrávania obrázka pomocou \emph{<input>} elementu.}
\end{figure}

\begin{enumerate}
	\item Užívateľ myšou klikne na~\emph{<input>} element.
	\item Vyskočí modálne okno, z~ktorého užívateľ vyberie obrázok.
	\item Užívateľ odošle formulár, obrázok je odoslaný na~server, kde je spracovaný.
\end{enumerate}

Výhody tohto riešenia sú predovšetkým:
\begin{itemize}
	\item Natívna podpora v~prehliadačoch - nie sú potrebné žiadne ďalšie JavaScriptové súbory.
	\item Jednoduché na~implementáciu.
	\item Podpora \emph{Drag\&Drop}.
\end{itemize}


\section{FormData a~XMLHttpRequest}

\emph{XMLHttpRequest Level 2} pridáva nové rozhranie \emph{FormData}, ktoré umožňuje vytvoriť reprezentáciu formulára v~tvare kľúč-hodnota, čím umožňuje zaslať formulár pomocou \emph{XMLHttpRequest} \cite{MDN_Formdata}. \emph{XMLHttpRequest} je API, ktoré umožňuje prenášať dáta medzi~prehliadačom a~serverom bez~toho, aby nastalo obnovenie stránky \cite{MDN_XMLHttpRequest}.

Vďaka \emph{FormData} je možné odstrániť obmedzenie, kde \emph{<input type="file"\textgreater} musí byť vložený do~formulára (<form\textgreater). Použitím \emph{XMLHttpRequest} eventu \emph{onprogress} je možné sledovať proces nahrávania \cite{MDN_XMLHttpRequest_progress}. Scénar nahrávania môže vyzerať nasledovne:
\begin{enumerate}
	\item Užívateľ myšou klikne na~\emph{<input>} element.
	\item Vyskočí modálne okno, z~ktorého užívateľ vyberie obrázok.
	\item Obrázok sa odošle na~server, zatiaľ čo užívateľ sleduje proces nahrávania. Keď je obrázok plne nahraný a~spracovaný, odošle sa jeho odkaz naspäť užívateľovi do~prehliadača.
	\item Užívateľovi sa zobrazí nahraný obrázok. Pokiaľ s~ním nie je spokojný, postupuje od~bodu 1. Ak je spokojný, odošle formulár. V~tomto prípade nedochádza k~žiadnej manipulácii s~obrázkom.
\end{enumerate}

Výhody tohto riešenia sú predovšetkým:
\begin{itemize}
	\item Nie je potrebné umiestnenie vo formulári.
	\item Je možné zaslať obrázok na~pozadí, čo umožňuje využitie v~moderných webových aplikáciách, kde je obnovenie stránky nežiaduce.
	\item Počas nahrávania je možné zobraziť proces v~percentách.
	\item Po~nahratí obrázka na~server je možné obrázok užívateľovi zobraziť. Takto ho užívateľ vidí pred~potvrdením formulára.
	\item Podpora \emph{Drag\&Drop}.
\end{itemize}

\begin{figure}[!ht]
	\centering
	\begin{tikzpicture}[node distance=.8cm, start chain=going below,]
		\node[punktchain, join] (input) {\emph{<input>} element};
		\node[punktchain, join] (modal) {Modálne okno};
		\node[punktchain, join] (file-choose) {Zvolenie obrázka};
		\node[punktchain, join] (send-file) {Odoslanie obrázka na~server};
		\node[punktchain, join, dashed, on chain=going right] (server-processing) {Spracovanie na~serveri};
		\node[punktchain, join, on chain=going below left] (image-display) {Zobrazenie obrázka};
	\end{tikzpicture}
	
	\caption{Schéma nahrávania obrázka pomocou \emph{<input>} elementu v~kombinácií s~\emph{FormData} a~{XMLHttpRequest}.}
\end{figure}


\section{Riešenia založené na~Adobe Flash a~Microsoft Silverlight}

V dobe písania tejto práce boli stále veľmi rozšírené riešenia, založené na~technológii Adobe Flash (ďalej spomínaný už len ako Flash). Flash, pôvodne navrhnutý pre~tvorbu multimediálneho obsahu - vektorovej grafiky, animácií, hier pre~prehliadač - sa stal populárnym aj na~tvorbu väčších aplikácií v~prehliadači, pracovanie so~vstupm z~webovej kamery či mikrofónu, až po~prehrávanie videí a~zvuku.

O~niečo podobné sa pokúšal Microsoft s~technológiou Silverlight (ďalej len Silverlight). Obe technológie nepoužívajú Javascript, ale iné jazyky - v~prípade technológie Flash ide o~ActionScript a~v prípade Silverlight ide o~Visual Basic, C\#, Ruby alebo Python. \textbf{V~týchto technológiách by preto bolo možné napísať aj celú našu knižnicu}, avšak nemohli by sme dosiahnuť podporu mobilných zariadení a~obe technológie vyžadujú inštaláciu rozšírení tretích strán do~prehliadača.

Hlavným dôvodom neúspechu technológie Flash a~Silverlight sa stalo odmietnutie podpory od~firmy Apple (pozri \ref{sec:mobile-support}). V~súčasnej dobe nie je podporovaný na~mobilných zariadeniach s~operačným systémom iOS, Android a~Windows Phone. Samotné Adobe už prestáva ďalej Flash podporovať a~zameriava sa radšej na~HTML5\cite{Flash_dead}.