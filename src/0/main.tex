\chapter{Úvod}

Nahrávanie obrázkov na~internet je neoddeliteľnou súčasťou našich životov. V~roku 1997 konzorcium W3C vydalo štandard HTML 3.2 \cite{html32}, v~ktorom definuje spôsob nahrávania súborov (a teda aj obrázkov) pomocou nepárového tágu HTML \hbox{<input type="file"\textgreater}. Tento spôsob nahrávania obrázkov pretrváva dodnes, avšak požiadavky sa zmenili. Snaha vývojárov zobrazovať nahraný obrázok pred~odoslaním formulára vyžadovala využívanie rôznych trikov (napríklad vkladanie neviditeľných rámcov) alebo iných technológií (Flash, Silverlight). Spomínané spôsoby väčšinou fungujú na~princípe, že sa na~pozadí obrázok nahraje a~následne sa pošle naspäť na~stránku, kde sa zobrazí. HTML5 prináša nové možnosti, ako nahrávať obrázky na~server.

Cieľom tejto práce je vytvoriť knižnicu v~jazyku JavaScript, ktorá by zobrazila obrázok pred~jeho odoslaním na~server. S~týmto obrázkom má byť možné posúvať, a~nastaviť tak jeho orezanie. Na~server sa bude odoslať už iba jeho zmenšená a~orezaná časť. Okrem~toho bude naša naša knižnica podporovať mobilné zariadenia a~tiež bude podporovať nahrávanie viacerých obrázkov. Súčasťou riešenia bude tiež dokumentácia, ktorá vysvetlí, ako knižnicu správne používať. Na~záver bude riešenie porovnané s~už existujúcimi riešeniami podobného zamerania.